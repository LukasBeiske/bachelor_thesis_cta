\thispagestyle{plain}

\section*{Abstract}
In recent years, gamma-ray astronomy provided many interesting results using Imaging Air Cherenkov Telescopes (IACTs).
To continue this trend, the Cherenkov Telescope Array (CTA) will be build as the next generation IACT experiment.
Currently the LST-1 prototype for the largest of the telescopes used by CTA is being tested at the Roque de los Muchachos observatory on La Palma, Spain.

In this work, I present a source independent high-level analysis for the LST-1 prototype based on the random forest algorithm.
For the training and application of the machine learning models, the aict-tools package is used.
The performance of the background separation, energy estimation and origin reconstrcution using the disp method is compared for 
different event pre-selection criteria.
It is shown that strict pre-selection is necessary to achieve good results and a decrease in performance for energies below $\SI{300}{\giga\electronvolt}$
is visible.
The impact of a mismatch between simulated and real optical efficiency of the camera is examined as well.

In the end a detection of the Crab Nebula with a significance of $\num{27.7} \sigma$ is achieved using data obtained from $\SI{3.96}{\hour}$ of observation time.
Additionally, a detection of the blazar Markarian 421 with a significance of $\num{43.7} \sigma$ is achieved based on $\SI{2.22}{\hour}$ of observation time.
