\chapter{Appendix}

\section{Reconstruction of main shower axis angle}
\begin{figure}
    \centering
    \begin{subfigure}{0.49\textwidth}
        \centering
        \includegraphics[width=\linewidth]{build/plots_disp/plot_7.pdf}
        \caption{\texttt{lstchain v0.5.2} and \texttt{intensity > 150}.}
        \label{fig:newMC_150}
    \end{subfigure}
    \hfill
    \begin{subfigure}{0.49\textwidth}
        \centering
        \includegraphics[width=\linewidth]{HDD/build_scaling_300/plots_disp/plot_7.pdf}
        \caption{\texttt{lstchain v0.5.2} and \texttt{intensity > 300}.}
        \label{fig:newMC_300}
    \end{subfigure}
    \newline\vfill
    \begin{subfigure}{0.49\textwidth}
        \centering
        \includegraphics[width=\linewidth]{HDD/build_noscaling_300/plots_disp/plot_7.pdf}
        \caption{\texttt{lstchain v0.5.1} and \texttt{intensity > 300}.}
        \label{fig:newMC_300}
    \end{subfigure}
    \caption{A histogram of the difference between the reconstructed main shower axis angle $\delta$ and the true simulated main shower axis angle $\delta_\text{true}$.
        In all three cases the reconstruction works well, as a difference of $\num{0}$ and $\pm \pi$ means that the angle was reconstructed correctly.
    }
    \label{fig:delta_comparison}
\end{figure}