\chapter{Conclusion and Outlook}
This work shows that a source independent approach for the high-level event reconstruction for the LST-1 prototype yields good results, 
if sufficiently strict event pre-selection is performend.

Necessary preprocessing was performend using two different versions of the official low-level analysis pipeline \texttt{lstchain}, which is still 
in development, and the different results were compared.
In this work, the high-level reconstruction is based on the random forest algorithm and its implementation in the \texttt{aict-tools} package.
In order to use the \texttt{aict-tools}, some changes to the data had to be done complementing the progressing development of the \texttt{aict-tools}.
By the time of writing many CTA conventions are now being suppored by the \texttt{aict-tools} and development continues.

Simulations of diffuse protons, diffuse gamma-rays and gamma-rays from a point-like source were used for the training of the reconstruction models and
the angular resolution was calculated using point-like gamma-rays.
Background separation, energy estimation and origin reconstruction based on the disp method perform well and the best angular resolution is 
achieved for energies between $\SI{1}{\tera\electronvolt}$ and $\SI{10}{\tera\electronvolt}$.
The trained models were used to analyze observations of the Crab Nebula and Markarian 421 done by the LST-1.
In both cases a detection with a significance of $\num{27.7} \sigma$ for the Crab Nebula and $\num{43.7} \sigma$ for Markarian 421 was achieved.

During the course of this work, members of the LST Collaboration discovered, that the optical efficiency of the camera is lower than 
the value assumed for the simulations.
To compensate for this, they introduced a constant scaling factor for the optical efficiency into the preprocessing.
This improved the results overall which is shown as part of this work.
If a less strict event pre-selection is used which includes more low energy events, the performance of the individual reconstruction models gets only slightly worse.
However this does not apply for the observational results for the Crab Nebula and Markarian 421, instead a much larger negative impact can be observed.

A steep decrease of the reconstruction performance is visible for energies below $\SI{0.3}{\tera\electronvolt}$.
Because of this, the observational results for less strict event pre-selection might improve if the size of the on and off regions $\theta_\text{max}^2$ and the
prediction threshold $t_\gamma$ get chosen differently depending on the reconstructed energy of the observed particle.
As the mismatch of the optical efficiency is fixed now, those two parameters should also be optimized based on simulations instead of the observational data 
for the Crab Nebula.
This would make the results for different observations more comparable.

As the LST-1 still was not fully functional in January, no observations of the Crab Nebula using wobble have been done by the time of writing.
Such observations would likely yield better results, because wobble mode enables better background estimation as described in \autoref{sec:wobble}.
Further optimization of the event pre-selection is also advised.
Overall the continued development of the whole analysis pipeline and new observations will help to improve on the results presented in this work.  