\chapter{Appendix}

\section{Reconstruction of the main shower axis angle}
\begin{figure}
    \centering
    \begin{subfigure}{0.49\textwidth}
        \centering
        \includegraphics[width=\linewidth]{HDD/build_scaling_300/plots_disp/plot_7.pdf}
        \caption{\texttt{lstchain v0.5.2} and criteria set one.}
    \end{subfigure}
    \hfill
    \begin{subfigure}{0.49\textwidth}
        \centering
        \includegraphics[width=\linewidth]{build/plots_disp/plot_7.pdf}
        \caption{\texttt{lstchain v0.5.2} and criteria set two.}
    \end{subfigure}
    \newline\vfill
    \begin{subfigure}{0.49\textwidth}
        \centering
        \includegraphics[width=\linewidth]{HDD/build_noscaling_300/plots_disp/plot_7.pdf}
        \caption{\texttt{lstchain v0.5.1} and criteria set one.}
    \end{subfigure}
    \hfill
    \begin{subfigure}{0.49\textwidth}
        \centering
        \includegraphics[width=\linewidth]{HDD/build_noscaling/plots_disp/plot_7.pdf}
        \caption{\texttt{lstchain v0.5.1} and criteria set two.}
    \end{subfigure}
    \caption{A histogram of the difference between the reconstructed main shower axis angle $\delta$ and the true simulated main shower axis angle $\delta_\text{true}$.
        In all three cases the reconstruction works well, as a difference of $\num{0}$ and $\pm \pi$ means that the angle was reconstructed correctly.
    }
    \label{fig:delta_comparison}
\end{figure}


%%%%%%%%%%%%%%%%%%%%%%%%%%%%%%%%%%%%%%%%%%%%%%%%%%%%%%%%%%%%%%%%%%%%%%%%%%%%%%%%%%%%%%%%%%%%%%%%%%%%%%%%%%%%%%%%%%%%%%%%%%%%%%%%%%%%%%%%%%%%%%%%%%%%%%%%%%%%%%%
\section{Plots for \texttt{lstchain v0.5.2} using the pre-selection criteria set one}
\begin{figure}
    \centering
    \includegraphics[width=0.7\textwidth]{HDD/build_scaling_300/plots_separator/plot_4.pdf}
    \caption{Feature importance for the background separation model trained on the \texttt{lstchain v0.5.2} simulations using the pre-selection criteria set one.
        The same plot for the model using the pre-selection criteria set two can be seen in \autoref{fig:separator_newMC_150_feature}.
    }
    \label{fig:separator_newMC_300_feature}
\end{figure}
\begin{figure}
    \centering
    \includegraphics[width=0.9\textwidth]{HDD/build_scaling_300/plots_crab/plot_1.pdf}
    \caption{Theta-Squared-Plot for the Crab Nebula with $\alpha = \sfrac{t_\text{on}}{t_\text{off}}$.
        The machine learning models were trained using the \texttt{lstchain v0.5.2} simulations and the pre-selection criteria set one.
        It is visible that the scaling of the event count for the off region is incorrect.
    }
    \label{fig:crab_best_total_time}
\end{figure}


%%%%%%%%%%%%%%%%%%%%%%%%%%%%%%%%%%%%%%%%%%%%%%%%%%%%%%%%%%%%%%%%%%%%%%%%%%%%%%%%%%%%%%%%%%%%%%%%%%%%%%%%%%%%%%%%%%%%%%%%%%%%%%%%%%%%%%%%%%%%%%%%%%%%%%%%%%%%%%%
\section{Plots for \texttt{lstchain v0.5.1} using the pre-selection criteria set one}

\begin{figure}
    \centering
    \begin{subfigure}{0.49\textwidth}
        \centering
        \includegraphics[width=\textwidth]{HDD/build_noscaling_300/plots_regressor/plot_1.pdf}
        \caption{Energy migration matrix.}
    \end{subfigure}
    \hfill
    \begin{subfigure}{0.49\textwidth}
        \centering
        \includegraphics[width=\textwidth]{HDD/build_noscaling_300/plots_regressor/plot_4.pdf}
        \caption{Feature importance.}
    \end{subfigure}
    \newline\vfill
    \begin{subfigure}{0.7\textwidth}
        \centering
        \includegraphics[width=\textwidth]{HDD/build_noscaling_300/plots_regressor/plot_3.pdf}
        \caption{Resolution and bias.}
    \end{subfigure}
    \caption{Performance and feature importance of the energy estimator trained on the \texttt{lstchain v0.5.1} simulation and the event pre-selection criteria set one.}
    \label{fig:regressor_oldMC_300}
\end{figure}

\begin{figure}
    \centering
    \begin{subfigure}{0.49\textwidth}
        \centering
        \includegraphics[width=\textwidth]{HDD/build_noscaling_300/plots_disp/plot_1.pdf}
        \caption{Migration matrix for \texttt{|disp|}.}
    \end{subfigure}
    \hfill
    \begin{subfigure}{0.49\textwidth}
        \centering
        \includegraphics[width=\textwidth]{HDD/build_noscaling_300/plots_disp/plot_3.pdf}
        \caption{ROC for \texttt{sign} with mean AUC of $\num{0.9878(5)}$.}
    \end{subfigure}
    \newline\vfill
    \begin{subfigure}{0.49\textwidth}
        \centering
        \includegraphics[width=\textwidth]{HDD/build_noscaling_300/plots_disp/plot_6.pdf}
        \caption{Feature importance for \texttt{|disp|}.}
    \end{subfigure}
    \hfill
    \begin{subfigure}{0.49\textwidth}
        \centering
        \includegraphics[width=\textwidth]{HDD/build_noscaling_300/plots_disp/plot_5.pdf}
        \caption{Feature importance for \texttt{sign}.}
    \end{subfigure}
    \newline\vfill
    \begin{subfigure}{0.7\textwidth}
        \centering
        \includegraphics[width=\textwidth]{HDD/build_noscaling_300/plots_disp/plot_8.pdf}
        \caption{Energy dependent performance of \texttt{|disp|} and \texttt{sign}.}
    \end{subfigure}
    \caption{Performance and feature importance of the origin reconstruction based on the \texttt{lstchain v0.5.1} simulation and the event pre-selection criteria set one.}
    \label{fig:origin_oldMC_300}
\end{figure}

\begin{figure}
    \centering
    \includegraphics[width=0.7\textwidth]{HDD/build_noscaling_300/plots_crab/plot_9.pdf}
    \caption{Angular resolution calculated for the \texttt{lstchain v0.5.1} simulation using the event pre-selection criteria set one.
        Each bin contains at least 50 events.
    }
    \label{fig:ang_res_oldMC_300}
\end{figure}

\begin{figure}
    \centering
    \begin{subfigure}{0.49\textwidth}
        \centering
        \includegraphics[width=\textwidth]{HDD/build_noscaling_300/plots_crab/plot_2.pdf}
        \caption{Crab Nebula, $S_\text{Li\&Ma} = \num{22.7} \sigma$.}
        \label{fig:crab_oldMC_300}
    \end{subfigure}
    \hfill
    \begin{subfigure}{0.49\textwidth}
        \centering
        \includegraphics[width=\textwidth]{HDD/build_noscaling_300/plots_mrk421/plot_2.pdf}
        \caption{Markarian 421, $S_\text{Li\&Ma} = \num{20.6} \sigma$.}
        \label{fig:mrk_oldMC_300}
    \end{subfigure}
    \caption{Theta-Squared-Plots for both observations using the reconstruction models trained on the \texttt{lstchain v0.5.1} simulations and using 
        the event pre-selection criteria set one.
    }
    \label{fig:obs_oldMC_300}
\end{figure}


%%%%%%%%%%%%%%%%%%%%%%%%%%%%%%%%%%%%%%%%%%%%%%%%%%%%%%%%%%%%%%%%%%%%%%%%%%%%%%%%%%%%%%%%%%%%%%%%%%%%%%%%%%%%%%%%%%%%%%%%%%%%%%%%%%%%%%%%%%%%%%%%%%%%%%%%%%%%%%%
\section{Plots for \texttt{lstchain v0.5.2} using the pre-selection criteria set two}
\begin{figure}
    \centering
    \begin{subfigure}{0.49\textwidth}
        \centering
        \includegraphics[width=\textwidth]{build/plots_separator/plot_1.pdf}
        \caption{ROC curve with mean AUC of $\num{0.9084(17)}$.}
        \label{fig:separator_newMC_150}
    \end{subfigure}
    \hfill
    \begin{subfigure}{0.49\textwidth}
        \centering
        \includegraphics[width=\textwidth]{build/plots_separator/plot_4.pdf}
        \caption{Feature importance.}
        \label{fig:separator_newMC_150_feature}
    \end{subfigure}
    \caption{Performance and feature importance of the background separation based on the \texttt{lstchain v0.5.2} simulations and the event pre-selection criteria set two.}
\end{figure}

\begin{figure}
    \centering
    \begin{subfigure}{0.49\textwidth}
        \centering
        \includegraphics[width=\textwidth]{build/plots_regressor/plot_1.pdf}
        \caption{Energy migration matrix.}
    \end{subfigure}
    \hfill
    \begin{subfigure}{0.49\textwidth}
        \centering
        \includegraphics[width=\textwidth]{build/plots_regressor/plot_4.pdf}
        \caption{Feature importance.}
    \end{subfigure}
    \newline\vfill
    \begin{subfigure}{0.7\textwidth}
        \centering
        \includegraphics[width=\textwidth]{build/plots_regressor/plot_3.pdf}
        \caption{Resolution and bias.}
        \label{fig:regressor_newMC_150_bias}
    \end{subfigure}
    \caption{Performance and feature importance of the energy estimator trained on the \texttt{lstchain v0.5.2} simulation and the event pre-selection criteria set two.}
    \label{fig:regressor_newMC_150}
\end{figure}

\begin{figure}
    \centering
    \begin{subfigure}{0.49\textwidth}
        \centering
        \includegraphics[width=\textwidth]{build/plots_disp/plot_1.pdf}
        \caption{Migration matrix for \texttt{|disp|}.}
    \end{subfigure}
    \hfill
    \begin{subfigure}{0.49\textwidth}
        \centering
        \includegraphics[width=\textwidth]{build/plots_disp/plot_3.pdf}
        \caption{ROC for \texttt{sign} with mean AUC of $\num{0.9775(16)}$.}
    \end{subfigure}
    \newline\vfill
    \begin{subfigure}{0.49\textwidth}
        \centering
        \includegraphics[width=\textwidth]{build/plots_disp/plot_6.pdf}
        \caption{Feature importance for \texttt{|disp|}.}
    \end{subfigure}
    \hfill
    \begin{subfigure}{0.49\textwidth}
        \centering
        \includegraphics[width=\textwidth]{build/plots_disp/plot_5.pdf}
        \caption{Feature importance for \texttt{sign}.}
    \end{subfigure}
    \newline\vfill
    \begin{subfigure}{0.7\textwidth}
        \centering
        \includegraphics[width=\textwidth]{build/plots_disp/plot_8.pdf}
        \caption{Energy dependent performance of \texttt{|disp|} and \texttt{sign}.}
        \label{fig:origin_newMC_150_e}
    \end{subfigure}
    \caption{Performance and feature importance of the origin reconstruction based on the \texttt{lstchain v0.5.2} simulation and the event pre-selection criteria set two.}
    \label{fig:origin_newMC_150}
\end{figure}

\begin{figure}
    \centering
    \includegraphics[width=0.7\textwidth]{build/plots_crab/plot_9.pdf}
    \caption{Angular resolution calculated for the \texttt{lstchain v0.5.2} simulation using the event pre-selection criteria set two.
        Each bin contains at least 50 events.
    }
    \label{fig:ang_res_newMC_150}
\end{figure}

\begin{figure}
    \centering
    \begin{subfigure}{0.49\textwidth}
        \centering
        \includegraphics[width=\textwidth]{build/plots_crab/plot_2.pdf}
        \caption{Crab Nebula, $S_\text{Li\&Ma} = \num{11.4} \sigma$.}
        \label{fig:crab_newMC_150}
    \end{subfigure}
    \hfill
    \begin{subfigure}{0.49\textwidth}
        \centering
        \includegraphics[width=\textwidth]{build/plots_mrk421/plot_2.pdf}
        \caption{Markarian 421, $S_\text{Li\&Ma} = \num{9.9} \sigma$.}
        \label{fig:mrk_newMC_150}
    \end{subfigure}
    \caption{Theta-Squared-Plots for both observations using the reconstruction models trained on the \texttt{lstchain v0.5.2} simulations and using 
        the event pre-selection criteria set two.
    }
    \label{fig:obs_newMC_150}
\end{figure}



%%%%%%%%%%%%%%%%%%%%%%%%%%%%%%%%%%%%%%%%%%%%%%%%%%%%%%%%%%%%%%%%%%%%%%%%%%%%%%%%%%%%%%%%%%%%%%%%%%%%%%%%%%%%%%%%%%%%%%%%%%%%%%%%%%%%%%%%%%%%%%%%%%%%%%%%%%%%%%%
\section{Plots for \texttt{lstchain v0.5.1} using the pre-selection criteria set two}
\begin{figure}
    \centering
    \begin{subfigure}{0.49\textwidth}
        \centering
        \includegraphics[width=\textwidth]{HDD/build_noscaling/plots_separator/plot_1.pdf}
        \caption{ROC curve with mean AUC of $\num{0.8595(11)}$.}
        \label{fig:separator_oldMC_150}
    \end{subfigure}
    \hfill
    \begin{subfigure}{0.49\textwidth}
        \centering
        \includegraphics[width=\textwidth]{HDD/build_noscaling/plots_separator/plot_4.pdf}
        \caption{Feature importance.}
        \label{fig:separator_oldMC_150_feature}
    \end{subfigure}
    \caption{Performance and feature importance of the background separation based on the \texttt{lstchain v0.5.1} simulations and the event pre-selection criteria set two.}
\end{figure}

\begin{figure}
    \centering
    \begin{subfigure}{0.49\textwidth}
        \centering
        \includegraphics[width=\textwidth]{HDD/build_noscaling/plots_regressor/plot_1.pdf}
        \caption{Energy migration matrix.}
    \end{subfigure}
    \hfill
    \begin{subfigure}{0.49\textwidth}
        \centering
        \includegraphics[width=\textwidth]{HDD/build_noscaling/plots_regressor/plot_4.pdf}
        \caption{Feature importance.}
    \end{subfigure}
    \newline\vfill
    \begin{subfigure}{0.7\textwidth}
        \centering
        \includegraphics[width=\textwidth]{HDD/build_noscaling/plots_regressor/plot_3.pdf}
        \caption{Resolution and bias.}
    \end{subfigure}
    \caption{Performance and feature importance of the energy estimator trained on the \texttt{lstchain v0.5.1} simulation and the event pre-selection criteria set two.}
    \label{fig:regressor_oldMC_150}
\end{figure}

\begin{figure}
    \centering
    \begin{subfigure}{0.49\textwidth}
        \centering
        \includegraphics[width=\textwidth]{HDD/build_noscaling/plots_disp/plot_1.pdf}
        \caption{Migration matrix for \texttt{|disp|}.}
    \end{subfigure}
    \hfill
    \begin{subfigure}{0.49\textwidth}
        \centering
        \includegraphics[width=\textwidth]{HDD/build_noscaling/plots_disp/plot_3.pdf}
        \caption{ROC for \texttt{sign} with mean AUC of $\num{0.9562(9)}$.}
    \end{subfigure}
    \newline\vfill
    \begin{subfigure}{0.49\textwidth}
        \centering
        \includegraphics[width=\textwidth]{HDD/build_noscaling/plots_disp/plot_6.pdf}
        \caption{Feature importance for \texttt{|disp|}.}
    \end{subfigure}
    \hfill
    \begin{subfigure}{0.49\textwidth}
        \centering
        \includegraphics[width=\textwidth]{HDD/build_noscaling/plots_disp/plot_5.pdf}
        \caption{Feature importance for \texttt{sign}.}
    \end{subfigure}
    \newline\vfill
    \begin{subfigure}{0.7\textwidth}
        \centering
        \includegraphics[width=\textwidth]{HDD/build_noscaling/plots_disp/plot_8.pdf}
        \caption{Energy dependent performance of \texttt{|disp|} and \texttt{sign}.}
    \end{subfigure}
    \caption{Performance and feature importance of the origin reconstruction based on the \texttt{lstchain v0.5.1} simulation and the event pre-selection criteria set two.}
    \label{fig:origin_oldMC_150}
\end{figure}

\begin{figure}
    \centering
    \includegraphics[width=0.7\textwidth]{HDD/build_noscaling/plots_crab/plot_9.pdf}
    \caption{Angular resolution calculated for the \texttt{lstchain v0.5.1} simulation using the event pre-selection criteria set two.
        Each bin contains at least 50 events.
    }
    \label{fig:ang_res_oldMC_150}
\end{figure}

\begin{figure}
    \centering
    \begin{subfigure}{0.49\textwidth}
        \centering
        \includegraphics[width=\textwidth]{HDD/build_noscaling/plots_crab/plot_2.pdf}
        \caption{Crab Nebula, $S_\text{Li\&Ma} = \num{14.4} \sigma$.}
        \label{fig:crab_oldMC_150}
    \end{subfigure}
    \hfill
    \begin{subfigure}{0.49\textwidth}
        \centering
        \includegraphics[width=\textwidth]{HDD/build_noscaling/plots_mrk421/plot_2.pdf}
        \caption{Markarian 421, $S_\text{Li\&Ma} = \num{11.6} \sigma$.}
        \label{fig:mrk_oldMC_150}
    \end{subfigure}
    \caption{Theta-Squared-Plots for both observations using the reconstruction models trained on the \texttt{lstchain v0.5.1} simulations and using 
        the event pre-selection criteria set two.
    }
    \label{fig:obs_oldMC_150}
\end{figure}


\section{Configuration of the aict-tools}
\label{sec:config}
\begin{verbatim}
seed: 0
true_energy_column: mc_energy
energy_unit: TeV

multiple_telescopes: False

n_cross_validations : 5

separator:
  classifier : |
    ensemble.RandomForestClassifier(
        n_estimators=100,
        max_features='sqrt',
        n_jobs=-1,
        max_depth=15,
        criterion='entropy',
    )

  # randomly sample the data if you dont want to use the whole set
  n_background: 50000
  n_signal: 50000

  # Define the name of the output column for the positive class.
  output_name: gammaness

  features:
    - intensity
    - log_intensity
    - intercept
    - time_gradient
    - length
    - width
    - skewness
    - kurtosis
    - n_islands
    - n_pixels
    - leakage1_intensity
    - leakage1_pixel
    - leakage2_intensity
    - leakage2_pixel
    - concentration_cog
    - concentration_core
    - concentration_pixel

  # Generate some features using pd.DataFrame.eval
  feature_generation:
    needed_columns:
      - width
      - length
      - intensity
    features:
      area: width * length * @pi
      intensity_area: intensity / (width * length * @pi)
      area_intensity_cut_var: (width * length * @pi) / log(intensity)**2

disp:
  disp_regressor : |
    ensemble.RandomForestRegressor(
        n_estimators=100,
        max_features='sqrt',
        n_jobs=-1,
        max_depth=20,
    )

  sign_classifier: |
    ensemble.RandomForestClassifier(
        n_estimators=100,
        max_features='sqrt',
        n_jobs=-1,
        max_depth=20,
    )

  coordinate_transformation: CTA

  source_az_column: mc_az
  source_az_unit: rad
  source_alt_column: mc_alt 
  source_alt_unit: rad
  pointing_az_column: az_tel
  pointing_az_unit: rad
  pointing_alt_column: alt_tel
  pointing_alt_unit: rad

  cog_x_column: x  
  cog_y_column: y  
  delta_column: psi
  delta_unit: rad
  focal_length_column: focal_length
  focal_length_unit: m

  # randomly sample the data if you dont want to use the whole set
  n_signal : 50000

  features:
    - intensity
    - log_intensity
    - width
    - length
    - n_pixels
    - leakage1_intensity
    - leakage1_pixel
    - leakage2_intensity
    - leakage2_pixel
    - r
    - kurtosis
    - skewness
    - concentration_cog
    - concentration_core
    - time_gradient
    - intercept

  feature_generation:
    needed_columns:
      - width
      - length
      - log_intensity
    features:
      elongation: width / length
      area: width * length * @pi
      log_intensity_area: log_intensity / (width * length * @pi)

energy:
  regressor : |
    ensemble.RandomForestRegressor(
      n_estimators=100,
      max_features='sqrt',
      n_jobs=-1,
      max_depth=12,
    )

  # randomly sample the data if you dont want to use the whole set
  n_signal: 50000

  # define the name of the regression target
  target_column: mc_energy
  log_target: true

  # Define the name of the variable you want estimate by regression.
  output_name: gamma_energy_prediction

  features:
    - intensity
    - log_intensity
    - length
    - width
    - n_islands
    - n_pixels
    - skewness
    - leakage1_intensity
    - leakage1_pixel
    - leakage2_intensity
    - leakage2_pixel
    - concentration_cog
    - concentration_core
    - concentration_pixel

  # Generate some features using pd.DataFrame.eval
  feature_generation:
    needed_columns:
      - width
      - length
      - intensity
    features:
      area: width * length * @pi
      intensity_area: intensity / (width * length * @pi)
\end{verbatim}
