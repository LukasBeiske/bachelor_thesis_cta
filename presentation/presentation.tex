% This example is meant to be compiled with lualatex or xelatex
% The theme itself also supports pdflatex
\PassOptionsToPackage{unicode}{hyperref}
\documentclass[aspectratio=1610, 9pt]{beamer}

% Load packages you need here
\usepackage{polyglossia}
\setmainlanguage{english}

\usepackage{csquotes}
    

\usepackage{amsmath}
\usepackage{amssymb}
\usepackage{mathtools}
\usepackage{siunitx}
\usepackage{booktabs}

\usepackage{pgfpages}
\setbeameroption{show notes on second screen}

\usepackage[
  style=numeric,
  sorting=none,
  backend=biber,
]{biblatex}
\addbibresource{references.bib}

\usepackage{hyperref}
\usepackage{bookmark}

\usepackage{xfrac}
\usepackage{tikz}
\usetikzlibrary{
  overlay-beamer-styles,
  calc,
  tikzmark,
  decorations.pathreplacing,
}

% load the theme after all packages

\usetheme[
  showtotalframes, % show total number of frames in the footline
]{tudo}

% Put settings here, like
\unimathsetup{
  math-style=ISO,
  bold-style=ISO,
  nabla=upright,
  partial=upright,
  mathrm=sym,
}

\newcommand{\roundpic}[6][]{
  \node [circle, draw, color=tugreen, minimum width = #2,
    path picture = {
      \node [#1] at (path picture bounding box.center) {
        \includegraphics[width=#3]{#4}};
    }] at (#5,#6) {};}


\title{High-level Event Reconstruction for the LST-1 Prototype of the Cherenkov Telescope Array}
\author[L.~Beiske]{Lukas Beiske}
\institute[E5b]{E5b Astroparticle Physics \\  Department of Physics - TU Dortmund}
\titlegraphic{\includegraphics[width=0.5\textwidth]{images/LST_2.jpg}}


\begin{document}

\maketitle

\begin{frame}{Overview}
  \tableofcontents
\end{frame}

\section{Introduction}
\chapter{Gamma ray astronomy}
In astronomy, observations can be conducted using many different particles emitted by cosmic sources. 
These particles include photons covering a wide energy spectrum from the lowest radio up until the highest gamma ray domain,
but neutrinos and particles carrying electrical charge like protons or nuclei can be observed as well (see \autoref{fig:mma}). 
Recently, it even became possible to detect gravitational waves emitted by very highly energetic events like the merging of two black holes \cite{PhysRevLett.116.061102}.
\begin{figure}
    \centering
    \includegraphics[width=0.8\textwidth]{images/cosmic_messengers.png}
    \caption{Only uncharged particles carry information about their origin when they arrive on Earth.
        Neutrinos are hard to detect as they only interact via the weak force.
        Photons on the other hand can be absorbed in gas clouds, but are detectable using satellite- and ground-based observatories \cite{mma_icecube}.
    }
    \label{fig:mma}
\end{figure}

Photons and neutrinos are of special interest as they are not influenced by cosmic electromagnetic fields and therefore it is possible to reconstruct their origin 
position and study their sources.
As high energy gamma radiation cannot be produced thermally, more complex processes involving charged particles (Cosmic Rays) have to be considered 
(Further information in \cite{s_funk}).

The most important source class within our own galaxy are supernova remnants like the Crab Nebula \cite{nuimeprn12618}. 
Because of its constant and high gamma ray flux the Crab Nebula is often used as a \enquote{standart candle} in gamma ray astronomy, observations of which are also 
analysed as part of this work.
Extragalactic sources for gamma rays are mostly supermassive black holes at the center of galaxies which accrete matter from their surroundings,
so called active galactic nuclei (AGN).
This results in the formation of a disk around the black hole and sometimes relativistic jets are emitted perpendicular to the disk.
AGNs are classified depending on if a jet is emitted, how bright they are and at which angle they are observed as shown in \autoref{fig:agn}.
One example of a blazar that is very close to Earth is Markarian 421 located at a redshift of $z = \num{0.03}$ \cite{Albert_2007}.
Observations of this source are analysed in \autoref{ch:results}.
\begin{figure}
    \centering
    \includegraphics[width=0.73\textwidth]{images/agn.png}
    \caption{AGNs consist of a supermassive black hole in the center surrounded by an accretion disc which is again surrounded by a gas torus.
        The classification of AGNs is based on three main characteristics, the existence of a jet, their brightness and the angle under which they can be observed \cite{doi:10.1002/9783527666829.ch4}.
    }
    \label{fig:agn}
\end{figure}


As Earth's atmosphere is not transparent for high energy gamma rays, direct observations of gamma rays can only be done by satellite based observatories 
like the Large Area Telescope on the \textit{Fermi} Gamma-Ray Space Telescope (\textit{Fermi}-LAT).
On the ground, indirect observations are possible by using Imaging Air Cherenkov Telescopes (IACTs) which will be explained further in \autoref{ch:cta}.


\AtBeginSection{
  \begin{frame}{Overview}
    \tableofcontents[currentsection]
  \end{frame}
}

\section{CTA and the LST-1 prototype}
\chapter{Imaging Air Cherenkov Telescopes and the Cherenkov Telescope Array}
\label{ch:cta}

\section{Imaging Air Cherenkov Telescopes}
For IACTs the atmosphere itself acts as the detector medium. 
If a high energy primary particle interacts with the atmosphere it starts a cascade of secondary paricles called air showers.
The charged secondary particles travel faster than the speed of light in air resulting in the emission of Cherenkov Light.
This light is emitted along the moving direction of the charged particle.
The Cherenkov Light is then collected by mirrors and projected onto a camera system which is able to record single photons with a time resolution of a few nanoseconds.

As both charged primary paricles and photons start air showers a dominant background of hadronic air showers is recorded by IACTs.
In hadronic air showers many different interactions are possible due to the strong force.
These varied interactions lead to the more complex shower pattern of hadronic air showers resulting in the possibility to distinguish them from purely electromagnetic
air showers caused by photons or electrons.
Electromagnetic air showers only consist of two processes. 
High energy photons produce electron-positron pairs and the total energy of those two paricles equals the photon energy. 
High energy electrons/positrons then generate photons again through bremsstahlung. 
These two processes continue until the photon energy falls under energy threshold for pair production of $\SI{1022}{\kilo\electronvolt}$.


\section{CTA and the LST-1 prototype}
The Cherenkov Telescope Array (CTA) aims to be the next generation IACT experiment by providing a sensitivity at least an order of magnitude better than current experiments.
It will be comprised of two sites, one in the northern and one in the southern hemisphere which will consist of differently sized telescopes. 
The smallest one will be the Small-Sized Telescope (SST) sensitive for the highest energies above $\SI{5}{\tera\electronvolt}$ up until $\SI{300}{\tera\electronvolt}$.
The Medium-Sized Telescope (MST) is most sensitive for energies between $\SI{150}{\giga\electronvolt}$ and $\SI{5}{\tera\electronvolt}$ and the 
Large-Sized Telescope with a mirror diameter of $\SI{23}{\meter}$ will cover the lowest energies from $\SI{20}{\giga\electronvolt}$ up until $\SI{150}{\giga\electronvolt}$.
A comparsion of the telescopes can be seen in \autoref{fig:telescopes}
\begin{figure}
    \centering
    \includegraphics[width=0.7\textwidth]{logos/CTA_telescopes.png}
    \caption{Telescopes \cite{cta}.}
    \label{fig:telescopes}
\end{figure}

The southern site will be located in the Atacama desert in Chile and will consist of \num{4} LSTs, \num{25} MSTs and \num{70} SSTs.
The northern site will be build as part of the Observatorio del Roque de los Muchachos (ORM) on LaPalma and will include \num{4} LSTs and \num{15} MSTs.
More information about the CTA and its scientific capabilities can be found in \cite{science_with_cta}.

***More info about the sites/SST, MST ?***

The ORM is also the location of the first prototype for the LSTs inaugurated on the 10 October 2018, the LST-1 which is the subject of this work \cite{lst_inauguration}.

***More info about the LST1***



\section{Data used and preprocessing}
\chapter{Preprocessing of the data}

\section{Reconstruction of physical properties using machine learning}
\begin{frame}{Event pre-selection}
    \begin{columns}[onlytextwidth, t]
        \begin{column}{0.45\textwidth}
            \begin{align*}
                \mathtt{intensity} &> 300\\
                \mathtt{leakage1\_intensity} &< 0.2\\
                \mathtt{leakage2\_intensity} &< 0.2
            \end{align*}
            \begin{itemize}
                \item[\textbf{\textcolor{green}{+}}] Remove incomplete shower images
                \item[\textbf{\textcolor{green}{+}}] Remove dim events
                \item[\textbf{\textcolor{red}{-}}] Remove low energy events
            \end{itemize}
        \end{column}
        \begin{column}{0.45\textwidth}
            \only<2>{
                \begin{align*}
                    \mathtt{intensity} &> 150\\
                    \mathtt{leakage1\_intensity} &< 0.2\\
                    \mathtt{leakage2\_intensity} &< 0.2
                \end{align*}
                \begin{itemize}
                    \item[\textbf{\textcolor{green}{+}}] Lower energy events included
                    \item[\textbf{\textcolor{red}{-}}] which are harder to reconstruct
                \end{itemize}
            }
        \end{column}
    \end{columns}

    \note<1->[item]{Zuerst Eventselektion $\to$ diese Kriterien}
    \note<1->[item]{\texttt{leakage} $\to$ entferne unvollständige Bilder}
    \note<1->[item]{\texttt{intensity} $\to$ entferne dunkle Bilder (hell besser rekonstruierbar)}
    \note<2->[item]{Derart großer Wert für \texttt{intensity} $\to$ entfernt niederenergetische Ereignisse
        \newline $\to$ Wert verringern \newline
    }
    \note<2->[item]{Plots im Folgenden: v0.5.2 + \texttt{intensity > 300}}
\end{frame}

\begin{frame}{The aict-tools}
    \begin{columns}[onlytextwidth]
        \begin{column}{0.65\textwidth}
            \begin{itemize}
                \item Apply event pre-selection
                \item Train and apply scikit-learn models for 
                    \begin{itemize}
                        \item Gamma-hadron separation
                        \item Energy estimation
                        \item Origin reconstruction
                    \end{itemize}
                \item Create performance plots
            \end{itemize}
            \medskip
            \onslide<2->{
                Commandline applications \& configuration using a single YAML-file
                \begin{itemize}
                    \item[\textbf{\textcolor{tugreen}{\to}}] automated pipeline using make \textbf{\textcolor{tugreen}{\to}} reproducibility
                \end{itemize}
                Initially developed for FACT 
                \begin{itemize}
                    \item Different conventions than CTA (e.g. camera frame definition, units)
                        \begin{itemize}
                            \item[\textbf{\textcolor{tugreen}{\to}}] Conversion of LST data necessary
                        \end{itemize}
                    \item Continued development adds more CTA support 
                        \begin{itemize}
                            \item[\textbf{\textcolor{tugreen}{\to}}] Only data-structure adjustment and some renaming necessary now
                        \end{itemize}
                \end{itemize}
            }
        \end{column}
        \begin{column}{0.33\textwidth}
            \centering
            \includegraphics[width=\textwidth]{images/python.png}
            \includegraphics[width=0.8\textwidth]{images/scikit.png}
            \includegraphics[width=\textwidth]{images/pandas.png}
        \end{column}
    \end{columns}
    \onslide<2->{
        \begin{center}
            \vspace{0.5\baselineskip}
            \small\fullcite{aict-tools}
        \end{center}
    }

    \note<1->[item]{Durchführung der Analyse: aict-tools}
    \note<1->[item]{- Eventselektion anwenden
        \newline - Modelle des ML trainieren \& anwenden 
        \newline - scikit-learn Modelle 
        \newline - für Teilchentypbestimmung, Energieschätzung und Richtungsrekonstruktion
        \newline - Performance plots
    }
    \note<2->[item]{Vorteil: Anwendung mit Kommandozeile \& Konfiguration mit 1 YAML-Datei
        \newline $\to$ Analyse automatisieren 
        \newline $\to$ Reproduzierbarkeit 
        \newline $\to$ + Hier: Vergleich Kombinationen
    }
    \note<2->[item]{Ursprünglich für FACT entwickelt 
        \newline $\to$ Andere Konventionen 
        \newline $\to$ LST Daten anpassen
        \newline $\to$ Während Arbeit: Weiterentwicklung 
        \newline $\to$ nur noch minimale Anpassungen nötig
    }
\end{frame}

\begin{frame}{The disp method}
    \begin{columns}[onlytextwidth]
        \begin{column}{0.475\textwidth}
            \begin{itemize}
                \setlength\itemsep{1em}
                \item Origin reconstruction is 2D regression task 
                \item Assuming the main shower axis is correctly reconstructed\\
                    \begin{itemize}
                        \item[\textbf{\textcolor{tugreen}{\to}}] Estimate distance along axis $\to$ 1D regession
                        \item[\textbf{\textcolor{tugreen}{\to}}] Decide which side of cog $\to$ classification
                    \end{itemize}
            \end{itemize}
        \end{column}
        \begin{column}{0.475\textwidth}
            \centering
            \includegraphics[width=\textwidth]{images/disp.png}\\[-0.5\baselineskip]
            \hspace{1.5cm}\href{https://github.com/MaxNoe/phd_thesis}{[Max Noethe, PhD thesis]}
        \end{column}
    \end{columns}

    \note[item]{Richtungsrekonstruktion aict-tools $\to$ disp Methode}
    \note[item]{Normalerweise 2D Regression}
    \note[item]{Annahme: Hauptachse korrekt rekonstruiert
        \newline $\to$ Abstand Quelle entlang Hauptachse
        \newline $\to$ + Entscheidung welche Seite von Schwerpunkt
    }
\end{frame}

\section{Model performance}
\begin{frame}{Gamma-hadron separation}
    \begin{columns}[onlytextwidth]
        \begin{column}{0.475\textwidth}
            \centering
            \includegraphics[width=\textwidth]{../HDD/build_scaling_300/plots_separator/plot_4.pdf}
        \end{column}
        \begin{column}{0.475\textwidth}
            \begin{itemize}
                \setlength\itemsep{1em}
                \item Random forest classifier
                    \begin{itemize}
                        \item Trained on diffuse gammas and protons
                        \item \texttt{gamma\_prediction} score $\in [0, 1]$
                        \item[\textbf{\textcolor{tugreen}{\to}}] Choose prediction threshold $t_\gamma$
                    \end{itemize}
                \item Differentiate gamma-hadron shower images
                    \begin{itemize}
                        \item[\textbf{\textcolor{tugreen}{\to}}] Features describing shower image are most important
                    \end{itemize}
            \end{itemize}
        \end{column}
    \end{columns}

    \note[item]{Teilchentypbestimmung 
        \newline $\to$ Random forest Algo. zur Klassifizierung
        \newline $\to$ Trainiert: diffuse Gammas und Protonen
        \newline $\to$ Jedes Ereignis: \texttt{gamma\_prediction} Wert 
        \newline $\to$ Schwellwert für \texttt{gamma\_prediction}
    }
    \note[item]{Entscheidungsbaum basierter Algo. $\to$ Wichtigkeit Bildparameter}
    \note[item]{Komplexe hadronische Schauer $\to$ komplexere Bilder 
        \newline $\to$ Bildparameter für Schauerform
    }
\end{frame}

\begin{frame}{Gamma-hadron separation}
    \begin{columns}[onlytextwidth]
        \begin{column}{0.5\textwidth}
            \centering
            \includegraphics[width=1.1\textwidth]{../HDD/build_scaling_300/plots_separator/plot_1.pdf}
        \end{column}
        \begin{column}{0.5\textwidth}
            \begin{itemize}
                \item Perfromance evaluation independent from prediction threshold $t_\gamma$
            \end{itemize}
            \begin{table}
                \caption{Mean area under ROC Curve.}
                \sisetup{table-format=0.3}
                \begin{tabular}{ S  S  S}
                    \toprule
                     & {lstchain v0.5.1} & {lstchain v0.5.2} \\
                    \midrule
                    \texttt{intensity > 300} & 0.907 & 0.944 \\
                    \texttt{intensity > 150} & 0.860 & 0.908 \\
                    \bottomrule
                \end{tabular}
            \end{table}
        \end{column}
    \end{columns}

    \note[item]{Perfromance Klassifizier unabhängig Wahl Schwellwert 
        \newline $\to$ Fläche unter ROC Kurve $\to$ Bild
    }
    \note[item]{Ergebnisse verschiedene Kombinationen $\to$ Tabelle
        \newline $\to$ v0.5.2 deutlich besser
    }
\end{frame}

%%%%%%%%%%%%%%%%%%%%%%%%%%%%%%%%%%%%%%%%%%%%%%%%%%%%%%%%%%%%%%%%%%%%%%%%%%%%%%%%%%%%%%%%%%%%%%%%%%%%%%%%%%%%
\begin{frame}{Energy estimation}
    \begin{columns}[onlytextwidth]
        \begin{column}{0.475\textwidth}
            \centering
            \includegraphics<1>[width=\textwidth]{../HDD/build_scaling_300/plots_regressor/plot_4.pdf}
            \includegraphics<2>[width=\textwidth]{../HDD/build_scaling_300/plots_regressor/plot_1.pdf}
        \end{column}
        \begin{column}{0.475\textwidth}
            \begin{itemize}
                \setlength\itemsep{1em}
                \item Random forest regressor
                \item Trained on point-like gammas
                \item Light content most important
            \end{itemize}
        \end{column}
    \end{columns}

    \note<1->[item]{Energieschätzung 
        \newline $\to$ Random Forest
        \newline $\to$ Trainiert: Punktquellen Gammas    
    }
    \note<1->[item]{Beitrag Bildparameter $\to$ Helligkeit beschriebende Parameter}
    \note<2->[item]{Mögleichkeit Performance Regressor darzustellen $\to$ Migrations Matrix}
\end{frame}

\begin{frame}{Energy estimation}
    \begin{columns}[onlytextwidth]
        \begin{column}{0.5\textwidth}
            \centering
            Bias
            \includegraphics[width=\textwidth]{/home/lukas/Bachelorarbeit/bachelor_thesis_cta/build/plot_talk_2.pdf}
        \end{column}
        \begin{column}{0.5\textwidth}
            \centering
            (Quantile) Resolution
            \includegraphics[width=\textwidth]{../build/plot_talk_3.pdf}
        \end{column}
    \end{columns}

    \note[item]{Verrzerrung für verschiedene Kombinationen:
        \newline $\to$ Niedrige E: Überschätzt, da nur helle Ereignisse
        \newline $\to$ Hohe E: Unterschätzt, da Schauer nicht ganz im Kamera
    }
    \note[item]{Auflösung: 
        \newline $\to$ Insgesamt besser für höhere Energien
    }
    \note[item]{\texttt{intensity > 300} Kombinationen schlechter bei niedrige E $\to$ wenige, helle Ereignisse}
\end{frame}

%%%%%%%%%%%%%%%%%%%%%%%%%%%%%%%%%%%%%%%%%%%%%%%%%%%%%%%%%%%%%%%%%%%%%%%%%%%%%%%%%%%%%%%%%%%%%%%%%%%%%%%%%%%%
\begin{frame}{Origin reconstruction}
    \begin{columns}[onlytextwidth]
        \begin{column}{0.475\textwidth}
            \begin{itemize}
                \setlength\itemsep{1em}
                \item Random forest regressor/ classifier
                \item Trained on diffuse gammas
                \item Timing parameters and skewness (3rd moment along main shower axis) very important
            \end{itemize}
        \end{column}
        \begin{column}{0.475\textwidth}
            \centering
            \includegraphics<1>[width=\textwidth]{../HDD/build_scaling_300/plots_disp/plot_6.pdf}
            \includegraphics<2>[width=\textwidth]{../HDD/build_scaling_300/plots_disp/plot_5.pdf}
        \end{column}
    \end{columns}

    \note[item]{Richtungsrekonstruktion: disp \& sign
        \newline $\to$ erneut Random Forest
        \newline $\to$ Trainiert: Diffuse Gammas 
    }
    \note[item]{Parameter anhand Ankunftszeiten und Schiefe (entlang Hauptachse) wichtig}
    \note<2->[item]{$\to$ besonders bei sign}
\end{frame}

\begin{frame}{Origin reconstruction: angular resolution}
    \begin{columns}[onlytextwidth]
        \begin{column}{0.5\textwidth}
            \centering
            v0.5.2 and intensity > 300
            \includegraphics[width=\textwidth]{../HDD/build_scaling_300/plots_crab/plot_9.pdf}
        \end{column}
        \begin{column}{0.5\textwidth}
            \centering
            correct sign and $p_\gamma > 0.6$
            \includegraphics[width=\textwidth]{../build/plot_talk_1.pdf}
        \end{column}
    \end{columns}

    \note[item]{Insgesamt Performance Richtungsrekonstruktion $\to$ Winkelauflösung
        \newline $\to$ Radius in dem 68\% der Ereignisse einer Punktquelle rekonstruiert
    }
    \note[item]{Sinnvolle Einschränkung 
        \newline $\to$ Nur events mit korrektem sign $\to$ sonst weit weg von echter Quelle
        \newline $\to$ + Ereignisse, die wahrsch. Gammas
    }
    \note[item]{Vergleich Kombinationen 
        \newline $\to$ deutliche Verbesserung mit v0.5.2
        \newline $\to$ \texttt{intensity > 300} besser als \texttt{intensity > 150}
    }
\end{frame}

\section{Results for observational data}
\begin{frame}{Detection of the Crab Nebula}
    \begin{columns}[onlytextwidth]
        \begin{column}{0.64\textwidth}
            \centering
            \includegraphics[width=\textwidth]{../HDD/build_scaling_300/plots_crab/plot_2.pdf}
        \end{column}
        \begin{column}{0.35\textwidth}
            \hspace{0.3cm}\textcolor{tugreen}{Date:} 18th January, 2020\\
            \hspace{0.3cm}\textcolor{tugreen}{ON:} \SI{2.6}{\hour}\\
            \hspace{0.3cm}\textcolor{tugreen}{OFF:} \SI{1.3}{\hour}\\
            \medskip
            \begin{itemize}
                \item Different observation times for ON and OFF $\to$ scaling necessary:
                    \begin{align*}
                        \alpha = \frac{N_\text{on}(\SI{0.5}{\degree\squared} < \theta^2 < \SI{1}{\degree\squared})}
                            {N_\text{off}(\SI{0.5}{\degree\squared} < \theta^2 < \SI{1}{\degree\squared})}
                    \end{align*}
                \item $\theta^2_\text{max}$ and $t_\gamma$ optimised for maximum detection significance
            \end{itemize}
        \end{column}
    \end{columns}

    \note[item]{Krebsnebel}
    \note[item]{Unterschieliche Beobachtungszeit $\to$ Ereignisraten skalieren
        \newline Alle Ereignisse $\SI{0.5}{\degree\squared} < \theta^2 < \SI{1}{\degree\squared}$ vergleichen
    }
    \note[item]{Radius $\theta^2_\text{max}$ \& Schwellwert $t_\gamma$ für Berechung festlegen $\to$ Detektionssignifikanz
        \newline $\to$ Signifikans von 27.7$\sigma$ 
        \newline $\to$ $\theta^2_\text{max}$ und $t_\gamma$ hier optimiert
    }
\end{frame}

\begin{frame}{Detection of Markarian 421}
    \begin{columns}[onlytextwidth]
        \begin{column}{0.64\textwidth}
            \centering
            \includegraphics[width=\textwidth]{../HDD/build_scaling_300/plots_mrk421/plot_2.pdf}
        \end{column}
        \begin{column}{0.35\textwidth}
            \hspace{0.3cm}\textcolor{tugreen}{Date:} 20th June, 2020\\
            \hspace{0.3cm}\textcolor{tugreen}{Wobble:} \SI{2.2}{\hour}\\
            \medskip
            \begin{itemize}
                \item $n = \num{5}$ OFF regions
            \end{itemize}
        \end{column}
    \end{columns}

    \note[item]{Wobble Mode von Markarian 421}
    \note[item]{n = 5 äquidistante OFF regionen entlang Kreis}
    \note[item]{$\theta^2_\text{max}$ und $t_\gamma$ wie Krebsnebel $\to$ Signifikanz von 43.7$\sigma$ erreicht}
\end{frame}

\section{Conclusion and Outlook}
\begin{frame}[t]{Conclusion and Outlook}
    \begin{columns}[onlytextwidth]
        \begin{column}[T]{0.43\textwidth}
            \begin{itemize}
                \setlength\itemsep{1em} %!!!!!!!!!!!!!!!!!!!!!!!!!!!!!!!!
                \item<1-> aict-tools achieve good results for high energies
                \item<1-> lstchain v0.5.2 improved results
                    \begin{itemize}
                        \item[\textbf{\textcolor{tugreen}{\to}}] Impact of other changes besides scaling of optical efficiency?
                    \end{itemize}
                \item<2-> Optimisation of event pre-selection
                \item<2-> Energy dependent $\theta^2_\text{max}$ and $t_\gamma$ cuts optimised on simulations
                \item<2-> Wobble mode observations of the Crab Nebula
            \end{itemize}
        \end{column}
        \begin{column}[T]{0.55\textwidth}
            \vspace{-1cm}
            \begin{tikzpicture}
                \roundpic[xshift=0cm,yshift=0cm]{7.8cm}{14cm}{images/LST_1.jpg}{2}{1.5}
                \node[circle, fill, color=white, minimum width = 4.2cm] at (-3.6,3.3){};
                \roundpic[xshift=-.2cm,yshift=-.2cm]{4.2cm}{4.25cm}{build/plot_talk_1.pdf}{-3.6}{3.3}
                \node[circle, fill, color=white, minimum width = 4.9cm] at (-1,0){};
                \roundpic[xshift=-.2cm,yshift=-.2cm]{4.9cm}{4.95cm}{../HDD/build_scaling_300/plots_crab/plot_2.pdf}{-1}{0}
            \end{tikzpicture}
        \end{column}
    \end{columns}

    \note[item]{Gute Ergebnisse für LST-1 mit aict-tools möglich}
    \note[item]{Verbesserung mit lstchain v0.5.2
        \newline $\to$ Andere Änderungen vgl. mit v0.5.1?
    }
    \note<2->[item]{Nur 3 Parameter für Eventselektion $\to$ Optimierbar!}
    \note<2->[item]{$\theta^2_\text{max}$ Radius und $t_\gamma$ Schwellwert besser auf Simulation optimieren!}
    \note<2->[item]{Krebsnebeldaten alt $\to$ neue (wobble!) Daten vermutlich besser!}
\end{frame}


\begin{frame}
  \centering 
  \Huge\color{tugreen} Backup
\end{frame}

\begin{frame}{Accuracy sign \& $r^2$-score |disp|}
    \begin{columns}[onlytextwidth]
        \begin{column}{0.49\textwidth}
            \centering
            \small v0.5.2 and intensity > 300\\
            \includegraphics[width=.6\textwidth]{../HDD/build_scaling_300/plots_disp/plot_8.pdf}\\
            \small v0.5.1 and intensity > 300\\
            \includegraphics[width=.6\textwidth]{../HDD/build_noscaling_300/plots_disp/plot_8.pdf} 
        \end{column}
        \begin{column}{0.49\textwidth}
            \centering
            \small v0.5.2 and intensity > 150\\
            \includegraphics[width=.6\textwidth]{../build/plots_disp/plot_8.pdf}\\
            \small v0.5.1 and intensity > 150\\
            \includegraphics[width=.6\textwidth]{../HDD/build_noscaling/plots_disp/plot_8.pdf}  
        \end{column}
    \end{columns} 
\end{frame}

\begin{frame}{Crab Nebula: v0.5.2 and intensity > 300}
    \begin{columns}[onlytextwidth]
        \begin{column}{0.65\textwidth}
            \centering
            \includegraphics[width=\textwidth]{../HDD/build_scaling_300/plots_crab/plot_1.pdf}
        \end{column}
        \begin{column}{0.3\textwidth}
            \begin{itemize}
                \item $\alpha = \frac{t_\text{on}}{t_\text{off}}$
            \end{itemize}   
        \end{column}
    \end{columns}  
\end{frame}

\begin{frame}{Crab Nebula: v0.5.2 and intensity > 150}
    \centering
    \includegraphics[width=0.65\textwidth]{../build/plots_crab/plot_2.pdf}
\end{frame}

\begin{frame}{Crab Nebula: v0.5.1 and intensity > 300}
    \centering
    \includegraphics[width=0.65\textwidth]{../HDD/build_noscaling_300/plots_crab/plot_2.pdf}
\end{frame}

\begin{frame}{Crab Nebula: v0.5.1 and intensity > 150}
    \centering
    \includegraphics[width=0.65\textwidth]{../HDD/build_noscaling/plots_crab/plot_2.pdf}
\end{frame}

\end{document}
