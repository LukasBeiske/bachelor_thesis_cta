\chapter{Imaging Air Cherenkov Telescopes and the Cherenkov Telescope Array}
\label{ch:cta}

\section{Imaging Air Cherenkov Telescopes}
For IACTs the atmosphere itself acts as the detector medium. 
If a high energy primary particle interacts with the atmosphere it starts a cascade of secondary paricles called air showers.
The charged secondary particles travel faster than the speed of light in air resulting in the emission of Cherenkov Light.
This light is emitted along the moving direction of the charged particle.
The Cherenkov Light is then collected by mirrors and projected onto a camera system which is able to record single photons with a time resolution of a few nanoseconds.

As both charged primary paricles and photons start air showers a dominant background of hadronic air showers is recorded by IACTs.
In hadronic air showers many different interactions are possible due to the strong force.
These varied interactions lead to the more complex shower pattern of hadronic air showers resulting in the possibility to distinguish them from purely electromagnetic
air showers caused by photons or electrons.
Electromagnetic air showers only consist of two processes. 
High energy photons produce electron-positron pairs and the total energy of those two paricles equals the photon energy. 
High energy electrons/positrons then generate photons again through bremsstahlung. 
These two processes continue until the photon energy falls under energy threshold for pair production of $\SI{1022}{\kilo\electronvolt}$.


\section{CTA and the LST-1 prototype}
The Cherenkov Telescope Array (CTA) aims to be the next generation IACT experiment by providing a sensitivity at least an order of magnitude better than current experiments.
It will be comprised of two sites, one in the northern and one in the southern hemisphere which will consist of differently sized telescopes. 
The smallest one will be the Small-Sized Telescope (SST) sensitive for the highest energies above $\SI{5}{\tera\electronvolt}$ up until $\SI{300}{\tera\electronvolt}$.
The Medium-Sized Telescope (MST) is most sensitive for energies between $\SI{150}{\giga\electronvolt}$ and $\SI{5}{\tera\electronvolt}$ and the 
Large-Sized Telescope with a mirror diameter of $\SI{23}{\meter}$ will cover the lowest energies from $\SI{20}{\giga\electronvolt}$ up until $\SI{150}{\giga\electronvolt}$.
A comparsion of the telescopes can be seen in \autoref{fig:telescopes}
\begin{figure}
    \centering
    \includegraphics[width=0.7\textwidth]{logos/CTA_telescopes.png}
    \caption{Telescopes \cite{cta}.}
    \label{fig:telescopes}
\end{figure}

The southern site will be located in the Atacama desert in Chile and will consist of \num{4} LSTs, \num{25} MSTs and \num{70} SSTs.
The northern site will be build as part of the Observatorio del Roque de los Muchachos (ORM) on LaPalma and will include \num{4} LSTs and \num{15} MSTs.
More information about the CTA and its scientific capabilities can be found in \cite{science_with_cta}.

***More info about the sites/SST, MST ?***

The ORM is also the location of the first prototype for the LSTs inaugurated on the 10 October 2018, the LST-1 which is the subject of this work \cite{lst_inauguration}.

***More info about the LST1***

