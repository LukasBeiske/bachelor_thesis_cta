\begin{frame}{Gamma-hadron separation}
    \begin{columns}[onlytextwidth]
        \begin{column}{0.475\textwidth}
            \centering
            \includegraphics[width=\textwidth]{../HDD/build_scaling_300/plots_separator/plot_4.pdf}
        \end{column}
        \begin{column}{0.475\textwidth}
            \begin{itemize}
                \setlength\itemsep{1em}
                \item Random forest classifier
                    \begin{itemize}
                        \item Trained on diffuse gammas and protons
                        \item \texttt{gamma\_prediction} score $\in [0, 1]$
                        \item[\textbf{\textcolor{tugreen}{\to}}] Choose prediction threshold $t_\gamma$
                    \end{itemize}
                \item Differentiate gamma-hadron shower images
                    \begin{itemize}
                        \item[\textbf{\textcolor{tugreen}{\to}}] Features describing shower image are most important
                    \end{itemize}
            \end{itemize}
        \end{column}
    \end{columns}

    \note[item]{Teilchentypbestimmung 
        \newline $\to$ Random forest Algo. zur Klassifizierung
        \newline $\to$ Trainiert: diffuse Gammas und Protonen
        \newline $\to$ Jedes Ereignis: \texttt{gamma\_prediction} Wert 
        \newline $\to$ Schwellwert für \texttt{gamma\_prediction}
    }
    \note[item]{Entscheidungsbaum basierter Algo. $\to$ Wichtigkeit Bildparameter}
    \note[item]{Komplexe hadronische Schauer $\to$ komplexere Bilder 
        \newline $\to$ Bildparameter für Schauerform
    }
\end{frame}

\begin{frame}{Gamma-hadron separation}
    \begin{columns}[onlytextwidth]
        \begin{column}{0.5\textwidth}
            \centering
            \includegraphics[width=1.1\textwidth]{../HDD/build_scaling_300/plots_separator/plot_1.pdf}
        \end{column}
        \begin{column}{0.5\textwidth}
            \begin{itemize}
                \item Perfromance evaluation independent from prediction threshold $t_\gamma$
            \end{itemize}
            \begin{table}
                \caption{Mean area under ROC Curve.}
                \sisetup{table-format=0.3}
                \begin{tabular}{ S  S  S}
                    \toprule
                     & {lstchain v0.5.1} & {lstchain v0.5.2} \\
                    \midrule
                    \texttt{intensity > 300} & 0.907 & 0.944 \\
                    \texttt{intensity > 150} & 0.860 & 0.908 \\
                    \bottomrule
                \end{tabular}
            \end{table}
        \end{column}
    \end{columns}

    \note[item]{Perfromance Klassifizier unabhängig Wahl Schwellwert 
        \newline $\to$ Fläche unter ROC Kurve $\to$ Bild
    }
    \note[item]{Ergebnisse verschiedene Kombinationen $\to$ Tabelle
        \newline $\to$ v0.5.2 deutlich besser
    }
\end{frame}

%%%%%%%%%%%%%%%%%%%%%%%%%%%%%%%%%%%%%%%%%%%%%%%%%%%%%%%%%%%%%%%%%%%%%%%%%%%%%%%%%%%%%%%%%%%%%%%%%%%%%%%%%%%%
\begin{frame}{Energy estimation}
    \begin{columns}[onlytextwidth]
        \begin{column}{0.475\textwidth}
            \centering
            \includegraphics<1>[width=\textwidth]{../HDD/build_scaling_300/plots_regressor/plot_4.pdf}
            \includegraphics<2>[width=\textwidth]{../HDD/build_scaling_300/plots_regressor/plot_1.pdf}
        \end{column}
        \begin{column}{0.475\textwidth}
            \begin{itemize}
                \setlength\itemsep{1em}
                \item Random forest regressor
                \item Trained on point-like gammas
                \item Light content most important
            \end{itemize}
        \end{column}
    \end{columns}

    \note<1->[item]{Energieschätzung 
        \newline $\to$ Random Forest
        \newline $\to$ Trainiert: Punktquellen Gammas    
    }
    \note<1->[item]{Beitrag Bildparameter $\to$ Helligkeit beschriebende Parameter}
    \note<2->[item]{Mögleichkeit Performance Regressor darzustellen $\to$ Migrations Matrix}
\end{frame}

\begin{frame}{Energy estimation}
    \begin{columns}[onlytextwidth]
        \begin{column}{0.5\textwidth}
            \centering
            Bias
            \includegraphics[width=\textwidth]{build/plot_talk_2.pdf}
        \end{column}
        \begin{column}{0.5\textwidth}
            \centering
            (Quantile) Resolution
            \includegraphics[width=\textwidth]{build/plot_talk_3.pdf}
        \end{column}
    \end{columns}

    \note[item]{Verrzerrung für verschiedene Kombinationen:
        \newline $\to$ Niedrige E: Überschätzt, da nur helle Ereignisse
        \newline $\to$ Hohe E: Unterschätzt, da Schauer nicht ganz im Kamera
    }
    \note[item]{Auflösung: 
        \newline $\to$ Insgesamt besser für höhere Energien
    }
    \note[item]{\texttt{intensity > 300} Kombinationen schlechter bei niedrige E $\to$ wenige, helle Ereignisse}
\end{frame}

%%%%%%%%%%%%%%%%%%%%%%%%%%%%%%%%%%%%%%%%%%%%%%%%%%%%%%%%%%%%%%%%%%%%%%%%%%%%%%%%%%%%%%%%%%%%%%%%%%%%%%%%%%%%
\begin{frame}{Origin reconstruction}
    \begin{columns}[onlytextwidth]
        \begin{column}{0.475\textwidth}
            \begin{itemize}
                \setlength\itemsep{1em}
                \item Random forest regressor/ classifier
                \item Trained on diffuse gammas
                \item Timing parameters and skewness (3rd moment along main shower axis) very important
            \end{itemize}
        \end{column}
        \begin{column}{0.475\textwidth}
            \centering
            \includegraphics<1>[width=\textwidth]{../HDD/build_scaling_300/plots_disp/plot_6.pdf}
            \includegraphics<2>[width=\textwidth]{../HDD/build_scaling_300/plots_disp/plot_5.pdf}
        \end{column}
    \end{columns}

    \note[item]{Richtungsrekonstruktion: disp \& sign
        \newline $\to$ erneut Random Forest
        \newline $\to$ Trainiert: Diffuse Gammas 
    }
    \note[item]{Parameter anhand Ankunftszeiten und Schiefe (entlang Hauptachse) wichtig}
    \note<2->[item]{$\to$ besonders bei sign}
\end{frame}

\begin{frame}{Origin reconstruction: angular resolution}
    \begin{columns}[onlytextwidth]
        \begin{column}{0.5\textwidth}
            \centering
            v0.5.2 and intensity > 300
            \includegraphics[width=\textwidth]{../HDD/build_scaling_300/plots_crab/plot_9.pdf}
        \end{column}
        \begin{column}{0.5\textwidth}
            \centering
            correct sign and $p_\gamma > 0.6$
            \includegraphics[width=\textwidth]{build/plot_talk_1.pdf}
        \end{column}
    \end{columns}

    \note[item]{Insgesamt Performance Richtungsrekonstruktion $\to$ Winkelauflösung
        \newline $\to$ Radius in dem 68\% der Ereignisse einer Punktquelle rekonstruiert
    }
    \note[item]{Sinnvolle Einschränkung 
        \newline $\to$ Nur events mit korrektem sign $\to$ sonst weit weg von echter Quelle
        \newline $\to$ + Ereignisse, die wahrsch. Gammas
    }
    \note[item]{Vergleich Kombinationen 
        \newline $\to$ deutliche Verbesserung mit v0.5.2
        \newline $\to$ \texttt{intensity > 300} besser als \texttt{intensity > 150}
    }
\end{frame}